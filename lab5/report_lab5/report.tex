\documentclass[a4paper,12pt]{extarticle}
\usepackage[utf8x]{inputenc}
\usepackage[T1,T2A]{fontenc}
\usepackage[russian]{babel}
\usepackage{hyperref}
\usepackage{indentfirst}
\usepackage{listings}
\usepackage{color}
\usepackage{here}
\usepackage{array}
\usepackage{multirow}
\usepackage{graphicx}

\usepackage{caption}
\renewcommand{\lstlistingname}{Программа} % заголовок листингов кода

\bibliographystyle{ugost2008ls}

\usepackage{listings}
\lstset{ %
extendedchars=\true,
keepspaces=true,
language=C,						% choose the language of the code
basicstyle=\footnotesize,		% the size of the fonts that are used for the code
numbers=left,					% where to put the line-numbers
numberstyle=\footnotesize,		% the size of the fonts that are used for the line-numbers
stepnumber=1,					% the step between two line-numbers. If it is 1 each line will be numbered
numbersep=5pt,					% how far the line-numbers are from the code
backgroundcolor=\color{white},	% choose the background color. You must add \usepackage{color}
showspaces=false				% show spaces adding particular underscores
showstringspaces=false,			% underline spaces within strings
showtabs=false,					% show tabs within strings adding particular underscores
frame=single,           		% adds a frame around the code
tabsize=2,						% sets default tabsize to 2 spaces
captionpos=t,					% sets the caption-position to top
breaklines=true,				% sets automatic line breaking
breakatwhitespace=false,		% sets if automatic breaks should only happen at whitespace
escapeinside={\%*}{*)},			% if you want to add a comment within your code
postbreak=\raisebox{0ex}[0ex][0ex]{\ensuremath{\color{red}\hookrightarrow\space}},
texcl=true,
inputpath=listings,                     % директория с листингами
}

\usepackage[left=2cm,right=2cm,
top=2cm,bottom=2cm,bindingoffset=0cm]{geometry}

%% Нумерация картинок по секциям
\usepackage{chngcntr}
\counterwithin{figure}{section}
\counterwithin{table}{section}

%%Точки нумерации заголовков
\usepackage{titlesec}
\titlelabel{\thetitle.\quad}
\usepackage[dotinlabels]{titletoc}

%% Оформления подписи рисунка
\addto\captionsrussian{\renewcommand{\figurename}{Рисунок}}
\captionsetup[figure]{labelsep = period}

%% Подпись таблицы
\DeclareCaptionFormat{hfillstart}{\hfill#1#2#3\par}
\captionsetup[table]{format=hfillstart,labelsep=newline,justification=centering,skip=-10pt,textfont=bf}

%% Путь к каталогу с рисунками
\graphicspath{{fig/}}


\begin{document}	% начало документа

% Титульная страница
\begin{titlepage}	% начало титульной страницы

	\begin{center}		% выравнивание по центру

		\large Санкт-Петербургский политехнический университет Петра Великого\\
		\large Институт компьютерных наук и технологий \\
		\large Кафедра компьютерных систем и программных технологий\\[6cm]
		% название института, затем отступ 6см
		
		\huge Телекоммуникационные технологии\\[0.5cm] % название работы, затем отступ 0,5см
		\large Отчет по лабораторной работе №6\\[0.1cm]
		\large Цифровая модуляция\\[5cm]

	\end{center}


	\begin{flushright} % выравнивание по правому краю
		\begin{minipage}{0.25\textwidth} % врезка в половину ширины текста
			\begin{flushleft} % выровнять её содержимое по левому краю

				\large\textbf{Работу выполнил:}\\
				\large Графов Д.И.\\
				\large {Группа:} 33531/2\\
				
				\large \textbf{Преподаватель:}\\
				\large Богач Н.В.

			\end{flushleft}
		\end{minipage}
	\end{flushright}
	
	\vfill % заполнить всё доступное ниже пространство

	\begin{center}
	\large Санкт-Петербург\\
	\large \the\year % вывести дату
	\end{center} % закончить выравнивание по центру

\thispagestyle{empty} % не нумеровать страницу
\end{titlepage} % конец титульной страницы

\vfill % заполнить всё доступное ниже пространство


% Содержание
% Содержание
\renewcommand\contentsname{\centerline{Содержание}}
\tableofcontents
\newpage




\section{Цель работы}
Изучение частотной и фазовой модуляции/демодуляции сигнала.

\section{Программа работы}
Сгенерировать однотональный сигнал низкой частоты.
Выполнить фазовую модуляцию/демодуляцию сигнала по закону u(t) = (Um cos(Ωt + ks(t)), используя встроенную функ- цию MatLab pmmod, pmdemod
Получить спектр модулированного сигнала.
Выполнить частотную модуляцию/демодуляциюи.\\


\section{Теоретическая информация}
Частотная модуляция — вид аналоговой модуляции, при котором информационный сигнал управляет частотой несущего колебания. По сравнению с амплитудной модуляцией здесь амплитуда остаётся постоянной.
ЧМ применяется для высококачественной передачи звукового (низкочастотного) сигнала в радиовещании (в диапазоне УКВ), для звукового сопровождения телевизионных программ, передачи сигналов цветности в телевизионном стандарте SECAM, видеозаписи на магнитную ленту, музыкальных синтезаторах.
Высокое качество кодирования аудиосигнала обусловлено тем, что в радиовещании при ЧМ применяется большая (по сравнению с шириной спектра сигнала АМ) девиация несущего сигнала, а в приёмной аппаратуре используют ограничитель амплитуды радиосигнала для устранения импульсных помех. Такая модуляция называется широкополосной ЧМ. В радиосвязи применяется узкополосная ЧМ с небольшой девиацией частоты несущего сигнала.

Фазовая модуляция — вид модуляции, при которой фаза несущего колебания изменяется прямо пропорционально информационному сигналу.
В случае, когда информационный сигнал является дискретным, то говорят о фазовой манипуляции. Возможна относительная фазовая манипуляция (ОФМ), если информация передается не в самой фазе, а в разности фаз соседних сигналов в последовательности. Хотя для сокращения занимаемой полосы частот манипуляция может производится не прямоугольным, а сглаженным импульсом, например, колоколообразным, приподнятым косинусом и др., но и в этом случае обычно говорят о манипуляции.
По характеристикам фазовая модуляция близка к частотной модуляции. В случае синусоидального модулирующего (информационного) сигнала, результаты частотной и фазовой модуляции совпадают.




\section{Ход выполнения работы}

\subsection{Листинг}

\lstinputlisting[
	label=code:lab12
	caption={lab2.py},% для печати символ '_' требует выходной символ '\'
]{lab2.py}
\parindent=1cm % командна \lstinputlisting сбивает параментры отступа

\subsection{Вывод программы}

\begin{lstlisting}
correlation method  fft
[0 0 0 1 0 2 0 2 1 2 1 1 0 0 1 0 1 0]
time =  0.003647021992946975

correlation method  direct
[0 0 0 1 0 2 0 2 1 2 1 1 0 0 1 0 1 0]
time =  2.1521991584450006e-05

correlation method  auto
[0 0 0 1 0 2 0 2 1 2 1 1 0 0 1 0 1 0]
time =  0.00011062600242439657

sync_mess = sig[6 : 14] =  [1 0 1]
package start =  6
package =  [0 1 1 1 0 0 0 0]
\end{lstlisting}

\section{Выводы}
В данной работе с помощью корреляции была найдена синхропосылка в сигнале. 
\end{document}
