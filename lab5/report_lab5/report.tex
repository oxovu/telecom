\include{settings}

\begin{document}	% начало документа

% Титульная страница
\include{titlepage}

% Содержание
\include{ToC}


\section{Цель работы}
Изучение частотной и фазовой модуляции/демодуляции сигнала.

\section{Программа работы}
Сгенерировать однотональный сигнал низкой частоты.
Выполнить фазовую модуляцию/демодуляцию сигнала по закону u(t) = (Um cos(Ωt + ks(t)), используя встроенную функ- цию MatLab pmmod, pmdemod
Получить спектр модулированного сигнала.
Выполнить частотную модуляцию/демодуляциюи.\\


\section{Теоретическая информация}
Частотная модуляция — вид аналоговой модуляции, при котором информационный сигнал управляет частотой несущего колебания. По сравнению с амплитудной модуляцией здесь амплитуда остаётся постоянной.
ЧМ применяется для высококачественной передачи звукового (низкочастотного) сигнала в радиовещании (в диапазоне УКВ), для звукового сопровождения телевизионных программ, передачи сигналов цветности в телевизионном стандарте SECAM, видеозаписи на магнитную ленту, музыкальных синтезаторах.
Высокое качество кодирования аудиосигнала обусловлено тем, что в радиовещании при ЧМ применяется большая (по сравнению с шириной спектра сигнала АМ) девиация несущего сигнала, а в приёмной аппаратуре используют ограничитель амплитуды радиосигнала для устранения импульсных помех. Такая модуляция называется широкополосной ЧМ. В радиосвязи применяется узкополосная ЧМ с небольшой девиацией частоты несущего сигнала.

Фазовая модуляция — вид модуляции, при которой фаза несущего колебания изменяется прямо пропорционально информационному сигналу.
В случае, когда информационный сигнал является дискретным, то говорят о фазовой манипуляции. Возможна относительная фазовая манипуляция (ОФМ), если информация передается не в самой фазе, а в разности фаз соседних сигналов в последовательности. Хотя для сокращения занимаемой полосы частот манипуляция может производится не прямоугольным, а сглаженным импульсом, например, колоколообразным, приподнятым косинусом и др., но и в этом случае обычно говорят о манипуляции.
По характеристикам фазовая модуляция близка к частотной модуляции. В случае синусоидального модулирующего (информационного) сигнала, результаты частотной и фазовой модуляции совпадают.




\section{Ход выполнения работы}

\subsection{Листинг}

\lstinputlisting[
	label=code:lab12
	caption={lab2.py},% для печати символ '_' требует выходной символ '\'
]{lab2.py}
\parindent=1cm % командна \lstinputlisting сбивает параментры отступа

\subsection{Вывод программы}

\begin{lstlisting}
correlation method  fft
[0 0 0 1 0 2 0 2 1 2 1 1 0 0 1 0 1 0]
time =  0.003647021992946975

correlation method  direct
[0 0 0 1 0 2 0 2 1 2 1 1 0 0 1 0 1 0]
time =  2.1521991584450006e-05

correlation method  auto
[0 0 0 1 0 2 0 2 1 2 1 1 0 0 1 0 1 0]
time =  0.00011062600242439657

sync_mess = sig[6 : 14] =  [1 0 1]
package start =  6
package =  [0 1 1 1 0 0 0 0]
\end{lstlisting}

\section{Выводы}
В данной работе с помощью корреляции была найдена синхропосылка в сигнале. 
\end{document}
