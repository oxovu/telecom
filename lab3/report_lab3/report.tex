\include{settings}

\begin{document}	% начало документа

% Титульная страница
\include{titlepage}

% Содержание
\include{ToC}


\section{Цель работы}
Изучить воздействие ФНЧ на тестовый сигнал с шумом.

\section{Программа работы}
Сгенерировать гармонический сигнал с шумом и синтезировать ФНЧ. Получить сигнал во временной и частотной областях до и после фильтрации. Сделать выводы о воздействии ФНЧ на спектр сигнала.\\


\section{Теоретическая информация}
Фильтр нижних частот (ФНЧ) — электронный или любой другой фильтр, эффективно пропускающий частотный спектр сигнала ниже некоторой частоты (частоты среза) и подавляющий частоты сигнала выше этой частоты. Степень подавления каждой частоты зависит от вида фильтра. \\ Идеальный фильтр нижних частот (sinc-фильтр) полностью подавляет все частоты входного сигнала выше частоты среза и пропускает без изменений все частоты ниже частоты среза. Переходной зоны между частотами полосы подавления и полосы пропускания не существует. Идеальный фильтр нижних частот может быть реализован лишь теоретически с помощью умножения спектра (преобразования Фурье) входного сигнала на прямоугольную функцию в частотной области, или, что даёт тот же эффект, свёртки сигнала во временной области с sinc-функцией. \\ Фильтр с конечной импульсной характеристикой (нерекурсивный фильтр, КИХ-фильтр) — один из видов электронных фильтров, характерной особенностью которого является ограниченность по времени его импульсной характеристики (с какого-то момента времени она становится точно равной нулю). Знаменатель передаточной функции такого фильтра — некая константа. \\ Фильтр с бесконечной импульсной характеристикой (рекурсивный фильтр, БИХ-фильтр) — электронный фильтр, использующий один или более своих выходов в качестве входа, то есть образует обратную связь.Основным свойством таких фильтров является то, что их импульсная переходная характеристика имеет бесконечную длину во временной области, а передаточная функция имеет дробно-рациональный вид. Такие фильтры могут быть как аналоговыми так и цифровыми.


\section{Ход выполнения работы}

\subsection{Листинг}

\lstinputlisting[
	label=code:lab3
	caption={lab3.py},% для печати символ '_' требует выходной символ '\'
]{lab3.py}
\parindent=1cm % командна \lstinputlisting сбивает параментры отступа

\subsection{Граффики}

\begin{figure}[H]
	\begin{center}
		\includegraphics[scale=0.7]{noise&sig}
		\caption{шум и сигнал} 
		\label{pic:noise&sig} % название для ссылок внутри кода
	\end{center}
\end{figure}

\begin{figure}[H]
	\begin{center}
		\includegraphics[scale=0.7]{filtered}
		\caption{результат фильтрации} 
		\label{pic:filtered} % название для ссылок внутри кода
	\end{center}
\end{figure}

\begin{figure}[H]
	\begin{center}
		\includegraphics[scale=0.7]{sector}
		\caption{спектры шума и отфильтрованного сигнала} 
		\label{pic:sector} % название для ссылок внутри кода
	\end{center}
\end{figure}


\section{Выводы}
В данной работе была произведенна фильтрация сигнала. Результат фильтрации очень близок к исходному сигналу. На граффике, иллюстрирующем спектры видно, что шумов стало меньше, особенно в области высоких частот. 
\end{document}
