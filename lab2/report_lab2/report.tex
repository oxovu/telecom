\include{settings}

\begin{document}	% начало документа

% Титульная страница
\include{titlepage}

% Содержание
\include{ToC}


\section{Цель работы}
Получить представление о спектрах телекоммуникационных сигналов.

\section{Программа работы}
– Для сигналов, построенных в лабораторной работе No1, выполните расчет преобразования Фурье. Перечислите свойства преобразования Фурье.\\
– С помощью функции корреляции найдите позицию синхропосылки [101] в сигнале [0001010111000010]. Получите пакет данных, если известно, что его длина составляет 8 бит без учета синхропосылки. Вычислите корреляцию прямым методом, воспользуйтесь алгоритмом быстрой корреляции, сравните время работы обоих алгоритмов.\\


\section{Теоретическая информация}
Ряд Фурье — представление функции f с периодом  \tau  в виде ряда\\
\begin{equation}\label{eq:fourierrow}
f(x) = \frac{a_0}{2} + \sum \limits_{k=1}^{\infty} A_k \cos \left( \frac{2 k \pi x}{\tau} - \alpha_k \right) 
\end{equation}\\
Дискретное преобразование Фурье является линейным преобразованием. Переводит вектор временных отсчетов в вектор спектральных отсчетов той же длинны. 
Преобразование Фурье сигнала является разложением по гармоническим функциям всех частот в диапозоне от $-\infty$ до $+\infty$. Позволяет при работе с сигналами осуществить частотно-временной переход. \\
Корреляция, и ее частный случай для центрированных сигналов – ковариация, является методом анализа сигналов. Корреляционный анализ дает возможность установить в сигналах (или в рядах цифровых данных сигналов) наличие определенной связи изменения значений сигналов по независимой переменной. В функциональном пространстве сигналов эта степень связи может выражаться в нормированных единицах коэффициента корреляции, т.е. в косинусе угла между векторами сигналов, и, соответственно, будет принимать значения от 1 (полное совпадение сигналов) до -1 (полная противоположность) и не зависит от значения (масштаба) единиц измерений.





\section{Ход выполнения работы}

\subsection{Листинг}

\lstinputlisting[
	label=code:lab12
	caption={lab2.py},% для печати символ '_' требует выходной символ '\'
]{lab2.py}
\parindent=1cm % командна \lstinputlisting сбивает параментры отступа

\subsection{Вывод программы}

\begin{lstlisting}
correlation method  fft
[0 0 0 1 0 2 0 2 1 2 1 1 0 0 1 0 1 0]
time =  0.003647021992946975

correlation method  direct
[0 0 0 1 0 2 0 2 1 2 1 1 0 0 1 0 1 0]
time =  2.1521991584450006e-05

correlation method  auto
[0 0 0 1 0 2 0 2 1 2 1 1 0 0 1 0 1 0]
time =  0.00011062600242439657

sync_mess = sig[6 : 14] =  [1 0 1]
package start =  6
package =  [0 1 1 1 0 0 0 0]
\end{lstlisting}

\section{Выводы}
В данной работе с помощью корреляции была найдена синхропосылка в сигнале. 
\end{document}
