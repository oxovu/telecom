\include{settings}

\begin{document}	% начало документа

% Титульная страница
\include{titlepage}

% Содержание
\include{ToC}


\section{Цель работы}
Познакомиться со средствами генерации и визуализации простых сигналов.

\section{Программа работы}
Промоделировать синусоидальный и прямоугольный сигналы с различными параметрами. Получить их спектры. Вывести на график.

\section{Теоретическая информация}
Аналитическое исследование поведения информационной системы основано на построении адекватной математической модели, отражающей характеристики элементов системы и возможные способы их взаимодействия. Понятие функции, как определенной зависимости величины y от величины – х, с математической записью в виде у(х) позволяет применять математический аппарат функций в качестве базовой основы построения моделей технических систем. Функции, служащие для описания реальных сигналов, всегда вещественны. Понятие "сигнал" широко используется в информационных системах и обычно обозначает физический процесс, который является материальным носителем информационного сообщения – изменение какого-либо параметра носителя (напряжения, частоты, фазы, мощности, интенсивности и т.п.) во времени, в пространстве или в зависимости от других аргументов служит для передачи информации.

\section{Ход выполнения работы}

\subsection{Листинг}

\lstinputlisting[
	label=code:lab1,
	caption={lab1.py},% для печати символ '_' требует выходной символ '\'
]{lab1.py}
\parindent=1cm % командна \lstinputlisting сбивает параментры отступа

\subsection{Граффики}

\begin{figure}[H]
	\begin{center}
		\includegraphics[scale=0.7]{sinus_signal}
		\caption{синусоидальный сигнал} 
		\label{pic:sinus_signal} % название для ссылок внутри кода
	\end{center}
\end{figure}

\begin{figure}[H]
	\begin{center}
		\includegraphics[scale=0.7]{sinus_spectrum}
		\caption{спектр синусоидального сигнала} 
		\label{pic:sinus_spectrum} % название для ссылок внутри кода
	\end{center}
\end{figure}

\begin{figure}[H]
	\begin{center}
		\includegraphics[scale=0.7]{rectangle_signal}
		\caption{прямоугольный сигнал} 
		\label{pic:rectangle_signal} % название для ссылок внутри кода
	\end{center}
\end{figure}

\begin{figure}[H]
	\begin{center}
		\includegraphics[scale=0.7]{rectangle_spectrum}
		\caption{спектр прямоугольного сигнала} 
		\label{pic:rectangle_spectrum} % название для ссылок внутри кода
	\end{center}
\end{figure}

\begin{figure}[H]
	\begin{center}
		\includegraphics[scale=0.7]{triangle_signal}
		\caption{треугольный сигнал} 
		\label{pic:triangle_signal} % название для ссылок внутри кода
	\end{center}
\end{figure}

\begin{figure}[H]
	\begin{center}
		\includegraphics[scale=0.7]{triangle_spectrum}
		\caption{спектр треугольного сигнала} 
		\label{pic:triangle_spectrum} % название для ссылок внутри кода
	\end{center}
\end{figure}

\section{Выводы}
В данной работе были промоделированны и получены спектры трех сигналов: синусоидального, треугольного и прямоугольного. На практике было установлено, что непрерывные сигналы имеют дискретный спектр, а переодические сигналы имеют непрерывный спектр.
\end{document}
