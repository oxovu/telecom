% !TeX spellcheck = ru_RU
\include{settings}
\usepackage{minted}

\begin{document}	% начало документа

% Титульная страница
\include{titlepage}

% Содержание
\setcounter{page}{2}
\include{ToC}


\section{Цель работы}
Изучение методов модуляции цифровых сигналов.

\section{Программа работы}
\begin{enumerate}
	\item Получить сигналы BPSK, PSK, OQPSK, genQAM, MSK, M-FSK модуляторов.
	\item Построить их сигнальные созвездия.
	\item Провести сравнение изученных методов модуляции цифровых сигналов.
	
\end{enumerate}

\section{Теоретическая информация}
Модуляция (лат. modulatio — размеренность, ритмичность) — процесс изменения одного или нескольких параметров модулируемого несущего сигнала при помощи модулирующего сигнала.

В цифровой модуляции аналоговый несущий сигнал модулируется цифровым битовым потоком.
Существуют три фундаментальных типа цифровой модуляции (или шифтинга) и один гибридный:
\begin{itemize}
	\item ASK – Amplitude shift keying (Амплитудная двоичная модуляция).
	\item FSK – Frequency shift keying (Частотая двоичная модуляция).
	\item PSK – Phase shift keying (Фазовая двоичная модуляция).
	\item ASK/PSK.
\end{itemize}

Упомяну, что существует традиция в русской терминологии радиосвязи использовать для модуляции цифровым сигналом термин «манипуляция».

\begin{figure}[H]
	\begin{center}
		\includegraphics[width=0.4\linewidth]{pics/modulations}
		\caption{Виды манипуляций}
		\label{fig:modulations}
	\end{center}
\end{figure}

В случае амплитудного шифтинга амплитуда сигнала для логического нуля может быть (например) в два раза меньше логической единицы.

Частотная модуляция похожим образом представляет логическую единицу интервалом с большей частотой, чем ноль.

Фазовый шифтинг представляет «0» как сигнал без сдвига, а «1» как сигнал со сдвигом.

Каждая из схем имеет свои сильные и слабые стороны.
\begin{itemize}
	\item ASK хороша с точки зрения эффективности использования полосы частот, но подвержена искажениям при наличии шума и недостаточно эффективна с точки зрения потребляемой мощности.
	\item FSK – с точностью до наоборот, энергетически эффективна, но не эффективно использует полосу частот.
	\item PSK – хороша в обоих аспектах.
	\item ASK/PSK – комбинация двух схем. Она позволяет еще лучше использовать полосу частот.
\end{itemize}

Самая простая PSK схема (показанная на рисунке) имеет собственное название — Binary phase-shift keying. Используется единственный сдвиг фазы между «0» и «1» — 180 градусов, половина периода.

Существуют также QPSK и 8-PSK:

QPSK использует 4 различных сдвига фазы (по четверти периода) и может кодировать 2 бита в символе (01, 11, 00, 10). 8-PSK использует 8 разных сдвигов фаз и может кодировать 3 бита в символе. 

Одна из частных реализаций схемы ASK/PSK которая называется QAM — Quadrature Amplitude Modulation (квадратурная амплитудная модуляция (КАМ). Это метод объединения двух AM-сигналов в одном канале. Он позваляет удвоить эффективную пропускную способность. В QAM используется две несущих с одинаковой частотой но с разницей в фазе на четверть периода (отсюда и возникает слово квадратура). Более высокие уровни QAM строятся по тому же принципы, что и PSK.

\newpage
\section{Ход выполнения работы}
Данная работа выполнялась с помощью библиотеки GNURadio.

GNU Radio — программный инструментарий, предоставляющий разработчикам программно-определяемых радиосистем «строительные блоки», обеспечивающие основные функции цифровой обработки сигналов.

Чтобы познакомиться с видами цифровой модуляции, были созданы 3 блок-схемы.

Приведём изображение одной из них в интерфейсе программы GNURadio.

Основне различие между ними - блок, следующий за Random Source - моделирующий.

\begin{figure}[H]
	\begin{center}
		\includegraphics[width=\linewidth]{pics/GNURadio}
		\caption{Интерфейс GNURadio}
		\label{fig:GNURadio}
	\end{center}
\end{figure}

\newpage
\section{Результаты работы}
Были получены сигнальные созвездия и спектральные изображения для разных видов модуляции.

\begin{figure}[!htb]
	\minipage{0.32\textwidth}
	\includegraphics[width=\linewidth]{pics/BPSK}
	\caption{BPSK}\label{fig:BPSK}
	\endminipage\hfill
	\minipage{0.32\textwidth}
	\includegraphics[width=\linewidth]{pics/QPSK}
	\caption{QPSK}\label{fig:QPSK}
	\endminipage\hfill
	\minipage{0.32\textwidth}%
	\includegraphics[width=\linewidth]{pics/8-PSK}
	\caption{8-PSK}\label{fig:8-PSK}
	\endminipage
\end{figure}

\begin{figure}[!htb]
	\minipage{0.5\textwidth}
	\includegraphics[width=\linewidth]{pics/QAM-4}
	\caption{QAM-4}\label{fig:QAM-4}
	\endminipage\hfill
	\minipage{0.5\textwidth}
	\includegraphics[width=\linewidth]{pics/QAM-16}
	\caption{QAM-16}\label{fig:QAM-16}
	\endminipage\hfill
\end{figure}

\begin{figure}[H]
	\begin{center}
		\includegraphics[width=\linewidth]{pics/GMSK}
		\caption{GMSK}
		\label{fig:GMSK}
	\end{center}
\end{figure}

Гауссовская частотная модуляция с минимальным сдвигом (англ. Gaussian Minimum Shift Keying (GMSK)) — вид частотой модуляции (манипуляции) с индексом модуляции равным 0.5, при которой последовательность из прямоугольных информационных импульсов проходит через гауссовский фильтр нижних частот.

Преимущество данного вида модуляции в том, что сигналы с GMSK имеют высокую скорость спада уровня внеполосных излучений, то есть меньше мешают другим пользователям эфира, чем сигналы с MSK. Как и сигналы с MSK, сигналы с GMSK имеют непрерывную фазу.

\newpage
\section{Выводы}
В ходе выполнения работы я ознакомился c основными возможностями библиотеки GNURadi. В рамках базовых возможностей библиотки были продемонстрированы основные виды цифровой модуляции.

Из полученных спектральных созвездий можно сделать вывод, что: 
\begin{itemize}
	\item Чем сложнее схема модуляции, тем более пагубное воздействие на нее оказывают искажения при передаче, и тем меньше расстояние от базовой станции, на котором сигнал может быть успешно принят.
	
	\item Теоретически возможны PSK и QAM схемы еще более высокого уровня, но на практике при их использовании возникает слишком большое количество ошибок.
\end{itemize}

\end{document}
