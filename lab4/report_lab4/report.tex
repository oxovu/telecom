\include{settings}

\begin{document}	% начало документа

% Титульная страница
\include{titlepage}

% Содержание
\include{ToC}


\section{Цель работы}
Изучение амплитудной модуляции/демодуляции сигнала.

\section{Программа работы}
Сгенерировать однотональный сигнал низкой частоты.

Выполнить амплитудную модуляцию (АМ) сигнала по закону u(t) = (1 + MUm cos Ωt) cos(ω0t + φ0). 

Получить спектр модулированного сигнала.

Выполнить модуляцию с подавлением несущей
u(t) = MUm cos(Ωt) cos(ω0t + φ0). Получить спектр.

Выполнить однополосную модуляцию:
U m 􏰀N
u(t) = Um cos(Ωt) cos(ω0t+φ0)+ 2 положив n=1

Выполнить синхронное детектирование и получить исходный однополосный сигнал.

Рассчитать КПД модуляции.\\


\section{Теоретическая информация}
Амплитудная модуляция — вид модуляции, при которой изменяемым параметром несущего сигнала является его амплитуда. \\ Пусть
S(t) — информационный сигнал, |S(t)|<1,
Uc(t) — несущее колебание.
Тогда \\амплитудно-модулированный сигнал Uam(t) может быть записан следующим образом:\\
Uam(t)=Uc(t)*[1+m*S(t)].\\
Здесь m — некоторая константа, называемая коэффициентом модуляции. Формула описывает несущий сигнал Uc(t), модулированный по амплитуде сигналом S(t) с коэффициентом модуляции m. Предполагается также, что выполнены условия:
|S(t)|<1, 0<m<=1.

Выполнение условий необходимо для того, чтобы выражение в квадратных скобках в всегда было положительным. Если оно может принимать отрицательные значения в какой-то момент времени, то происходит так называемая перемодуляция (избыточная модуляция). Простые демодуляторы (типа квадратичного детектора) демодулируют такой сигнал с сильными искажениями. 

Демодуляция — процесс, обратный модуляции колебаний, выделение информационного (модулирующего) сигнала из модулированного колебания высокой (несущей) частоты. 

 Одним из самых распространенных методов демодуляции амплитудно-модулированных сигналов является синхронное детектирование. При синхронном детектировании амплитудно-модулированный сигнал умножается на опорное немодулированное колебание с частотой несущего колебания, затем получившийся сигнал пропускается через фильтр нижних частот. В результате умножения получается сигнал, состоящий из двух слагаемых, первое из которых прямо пропорционально исходному модулирующему сигналу, а второе — амплитудно-модулированному сигналу с удвоенной несущей частотой. Второе слагаемое подавляет фильтр нижних частот, таким образом оставляется сигнал, прямо пропорциональный исходному информационному сигналу.





\section{Ход выполнения работы}

\subsection{Листинг}

\lstinputlisting[
	label=code:lab12
	caption={lab4.py},% для печати символ '_' требует выходной символ '\'
]{lab4.py}
\parindent=1cm % командна \lstinputlisting сбивает параментры отступа

\subsection{Граффики}

\begin{figure}[H]
	\begin{center}
		\includegraphics[scale=0.7]{1}
		\caption{амплитудная модуляция и демодуляция} 
		\label{pic:1} % название для ссылок внутри кода
	\end{center}
\end{figure}

\begin{figure}[H]
	\begin{center}
		\includegraphics[scale=0.7]{2}
		\caption{спектр промодулированного сигнала} 
		\label{pic:triangle_spectrum} % название для ссылок внутри кода
	\end{center}
\end{figure}

\begin{figure}[H]
	\begin{center}
		\includegraphics[scale=0.7]{3}
		\caption{модуляция и демодуляция с подавлением несущей} 
		\label{pic:triangle_spectrum} % название для ссылок внутри кода
	\end{center}
\end{figure}

\begin{figure}[H]
	\begin{center}
		\includegraphics[scale=0.7]{4}
		\caption{спектр промодулированного сигнала} 
		\label{pic:triangle_spectrum} % название для ссылок внутри кода
	\end{center}
\end{figure}

\begin{figure}[H]
	\begin{center}
		\includegraphics[scale=0.7]{5}
		\caption{однополосная модуляция} 
		\label{pic:triangle_spectrum} % название для ссылок внутри кода
	\end{center}
\end{figure}

\begin{figure}[H]
	\begin{center}
		\includegraphics[scale=0.7]{6}
		\caption{спектр промодулированного сигнала} 
		\label{pic:triangle_spectrum} % название для ссылок внутри кода
	\end{center}
\end{figure}

\section{Выводы}
В данной работе были рассмотренны амплитудная модуляция и демодуляция, модуляция с подавлением нусещей и однополосная модуляция, а так же их спектры. 
\end{document}
